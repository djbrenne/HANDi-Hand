%%%%%%%%%%%%%%%%%%%%%%%%%%%%%%%%%%%%%%%%%
% Dylan Brenneis - HANDi-Hand Assembly Manual
% 
% Drafted using the following template, modified for this manual:
%%%%%%%%%%%%%%%%%%%%%%%%%%%%%%%%%%%%%%%%%
% Masters/Doctoral Thesis 
% LaTeX Template
% Version 2.5 (27/8/17)
%
% This template was downloaded from:
% http://www.LaTeXTemplates.com
%
% Version 2.x major modifications by:
% Vel (vel@latextemplates.com)
%
% This template is based on a template by:
% Steve Gunn (http://users.ecs.soton.ac.uk/srg/softwaretools/document/templates/)
% Sunil Patel (http://www.sunilpatel.co.uk/thesis-template/)
%
% Template license:
% CC BY-NC-SA 3.0 (http://creativecommons.org/licenses/by-nc-sa/3.0/)
%
%%%%%%%%%%%%%%%%%%%%%%%%%%%%%%%%%%%%%%%%%

%----------------------------------------------------------------------------------------
%	PACKAGES AND OTHER DOCUMENT CONFIGURATIONS
%----------------------------------------------------------------------------------------

\documentclass[
11pt, % The default document font size, options: 10pt, 11pt, 12pt
oneside, % Two side (alternating margins) for binding by default, uncomment to switch to one side
english, % ngerman for German
% Single line spacing, alternatives: onehalfspacing or doublespacing
%draft, % Uncomment to enable draft mode (no pictures, no links, overfull hboxes indicated)
%nolistspacing, % If the document is onehalfspacing or doublespacing, uncomment this to set spacing in lists to single
%liststotoc, % Uncomment to add the list of figures/tables/etc to the table of contents
%toctotoc, % Uncomment to add the main table of contents to the table of contents
%parskip, % Uncomment to add space between paragraphs
%nohyperref, % Uncomment to not load the hyperref package
headsepline, % Uncomment to get a line under the header
chapterinoneline, % Uncomment to place the chapter title next to the number on one line
%consistentlayout, % Uncomment to change the layout of the declaration, abstract and acknowledgements pages to match the default layout
openany, % Comment out to have chapters, acknowledgements, etc. always start on a right-hand page (creates blank pages where necessary)
]{Manual} % The class file specifying the document structure

\usepackage[utf8]{inputenc} % Required for inputting international characters
\usepackage[T1]{fontenc} % Output font encoding for international characters
\usepackage{lipsum}
\usepackage{mathpazo} % Use the Palatino font by default
\usepackage{amsmath} % Math Package - db
\usepackage{pdfpages} % For picture quote page - db
\usepackage{siunitx}
\usepackage{float}
\usepackage[hidelinks]{hyperref}
\usepackage{caption}
\floatstyle{plaintop}
\restylefloat{table}
\usepackage{longtable}

%----------------------------------------------------------------------------------------
%	MARGIN SETTINGS
%----------------------------------------------------------------------------------------

\geometry{
	paper=letterpaper, % Change to letterpaper for US letter
	inner=2.5cm, % Inner margin
	outer=2.5cm, % Outer margin
	%bindingoffset=.5cm, % Binding offset
	top=2.5cm, % Top margin
	bottom=2.5cm, % Bottom margin
	head=21.986pt,
	%showframe, % Uncomment to show how the type block is set on the page
}

%----------------------------------------------------------------------------------------
%	MANUAL INFORMATION
%----------------------------------------------------------------------------------------

\manualtitle{HANDi-Hand Assembly Manual} % Your thesis title, this is used in the title and abstract, print it elsewhere with \ttitle
\author{Dylan J. A. \textsc{Brenneis}} % Your name, this is used in the title page and abstract, print it elsewhere with \authorname

\AtBeginDocument{
\hypersetup{pdftitle=\ttitle} % Set the PDF's title to your title
\hypersetup{pdfauthor=\authorname} % Set the PDF's author to your name
}

\begin{document}

\frontmatter % Use roman page numbering style (i, ii, iii, iv...) for the pre-content pages

\pagestyle{plain} % Default to the plain heading style until the thesis style is called for the body content

%----------------------------------------------------------------------------------------
%	TITLE PAGE
%----------------------------------------------------------------------------------------

\begin{titlepage}

\begin{figure}
\includegraphics[width=0.7\linewidth]{Figures/Handi hand Logo.png}
\end{figure}

\begin{flushright}
\null
\vfill
{\Huge \bfseries \ttitle\par}\vspace{0.6cm} % Manual title


\href{http://www.dylanbrenneis.ca}{Dylan J. A. Brenneis}\\%\authorname} % Author name - remove the \href bracket to remove the link
August 2020
\end{flushright}
\end{titlepage}


%----------------------------------------------------------------------------------------
%	INTRODUCTION PAGE
%----------------------------------------------------------------------------------------

\begin{abstract}
\setcounter{page}{2}
\addchaptertocentry{Introduction} % Add the abstract to the table of contents
The HANDi Hand is an open-source robotic platform specifically designed for machine learning research in prosthetic control. The inexpensive and easily modifiable hardware allows versatility for research studies, and the suite of sensors provides valuable information for machine learning and prosthetics research.

The open-source release provides all solid-modelling files, .stl files, Arduino code, and assembly instruction required to construct a fully functional HANDi Hand, and should also give the maker enough flexibility to make alterations to the design as necessary to suit their own needs. Both left and right hand versions are available. To contact the original designers, or to receive support for your build, please visit  \href{https://blincdev.ca/}{BLINCdev.ca}.

This assembly manual outlines all the information required to print and source parts, and assemble the HANDi Hand as currently designed. The hand takes an estimated 30 hours to build.

\vfill
\flushright{\textit{\small{The HANDi Hand was originally released in August 2017, as published in MEC '17}}}
\end{abstract}
%----------------------------------------------------------------------------------------
%	LIST OF CONTENTS/FIGURES/TABLES PAGES
%----------------------------------------------------------------------------------------

\tableofcontents % Prints the main table of contents

\listoftables % Prints the list of tables

\listoffigures % Prints the list of figures


%----------------------------------------------------------------------------------------
%	DEFINITIONS
%----------------------------------------------------------------------------------------
\newpage
\addchaptertocentry{Glossary of Terms}
\noindent\textbf{\Huge{Glossary of Terms}}
\vspace{1cm}

\noindent\textbf{Digits}: The digits are referred to by standard numbering, beginning with the thumb as D1 as shown in Figure~\ref{fig:finger_nums}.
\begin{figure}[H]
\centering
\includegraphics[width=0.3\linewidth]{Figures/Finger_Numbering.png}
\caption{Numbering scheme used for finger naming.}
\label{fig:finger_nums}
\end{figure}
\noindent\textbf{Joints}: The joints are named in accordance with Figure~\ref{fig:joint_nums}. The names are constructed first with a digit indicator (i.e. D2) followed by a joint indicator D, I or P, indicating distal, intermediate, or proximal respectively. Potentiometers are named for the joints that they measure. The digit D0 refers to thumb rotation.
\begin{figure}[H]
\centering
\includegraphics[width=0.3\linewidth]{Figures/Joint_Numbering.png}
\caption{Numbering scheme used for joint naming.}
\label{fig:joint_nums}
\end{figure}

\noindent\textbf{Finger Parts}: Each phalanx of the finger is made up of multiple parts. The part names are contrived according to the following convention:
\begin{enumerate}
\item Phalanx indicator. PP = Proximal Phalanx, IP = Intermediate Phalanx, DP = Distal Phalanx, MC = Metacarpal
\item Part position indicator. P = Proximal, D = Distal
\item Position modifier. There are sometimes multiple parts in the same location that must be differentiated by their function (pivot, main, lock, tip, etc).
\item Handedness indicator. R = Right Hand, L = Left Hand
\end{enumerate}
An example part name would be \textbf{IP-P Pivot R}, for the proximal portion of the intermediate phalanx corresponding to the pivot, for the right-hand version of the HANDi Hand.
\newpage


%----------------------------------------------------------------------------------------
%	MAIN CONTENT 
%----------------------------------------------------------------------------------------

\mainmatter % Begin numeric (1,2,3...) page numbering

\pagestyle{thesis} % Return the page headers back to the "thesis" style

% Include the steps of assembly as separate files from the Steps folder
% Uncomment the lines as you write the chapters

% Required Materials

\chapter{Required Materials}  % Main chapter title

\label {Required Materials} % For referencing the chapter elsewhere, use \ref{Required Materials} 

%----------------------------------------------------------------------------------------

% Define some commands to keep the formatting separated from the content 
\newcommand{\keyword}[1]{\textbf{#1}}
\newcommand{\tabhead}[1]{\textbf{#1}}
\newcommand{\code}[1]{\texttt{#1}}
\newcommand{\file}[1]{\texttt{\bfseries#1}}
\newcommand{\option}[1]{\texttt{\itshape#1}}

%----------------------------------------------------------------------------------------

\section{3D Printed Parts}

The 3D printed parts are designed to be printed in PLA without support material and without rafts, except where indicated. Parts are designed for the print tolerance of a MakerBot Replicator 2. Some filing may be necessary to ensure a smooth running fit between parts.

All files required for 3D printing can be accessed via \href{https://blincdev.ca/}{BLINCdev.ca}. The suggested print specifications for each part are found in Table \ref{3d_printed_parts_list}. The table lists all of the part sets that must be printed for a complete hand. In the event that a particular component is needed, the individual STL files can be found on  \href{https://blincdev.ca/}{BLINCdev.ca} in addition to the grouped parts in Table \ref{3d_printed_parts_list}.

\begin{table}[H]
\centering
\caption{3D Printed Parts Specifications}
\label{3d_printed_parts_list}
\begin{tabular}{|p{4cm}|p{4cm}|p{2cm}|p{2.5cm}|}
\hline
\textbf{Part Name} & \textbf{Print Specifications} & \textbf{Estimated Print Time} & \textbf{Est. Material Weight} \\ \hline
Dorsal Palm & 0.2mm layer, 30\% infill, print with raft & 6h 0m & 64 g \\ \hline
Ventral Palm & 0.2 mm layer, 10\% infill & 1h 55m & 22 g  \\ \hline
Thumb Screw Cap &0.2 mm layer, 10\% infill &0h 15m &2 g  \\ \hline
D2 Full Finger &0.2 mm layer, 10\% infill &1h 10m &9 g  \\ \hline
D3 Full Finger &0.2 mm layer, 10\% infill &1h 10m &9 g  \\ \hline
D4 Full Finger &0.2 mm layer, 10\% infill &1h 10m &9 g  \\ \hline
D5 Full Finger &0.2 mm layer, 10\% infill &1h 10m &9 g  \\ \hline
Full Thumb &0.2 mm layer, 10\% infill &2h 30m &25 g  \\ \hline
Breadboard &0.1 mm layer, 10\% infill &0h 25m &3 g  \\ \hline
Connector Hub &0.2 mm layer, 10\% infill &0h 15m &3 g  \\ \hline
Pot Activator Set of 15 &0.1 mm layer, 10\% infill &0h 10m &1 g  \\ \hline
Pot Placeholder Set of 6 &0.2 mm layer, 10\% infill &0h 10m &2 g  \\ \hline
Servo Spool Full Set &0.1 mm layer, 10\% infill &0h 25m &4 g  \\ \hline
Servo Spur Gear &0.1 mm layer, 10\% infill &0h 5m &1 g  \\ \hline
\textbf{TOTAL:}&\centering{-}  & 16h 50m & 163 g \\ \hline
\end{tabular}
\end{table} 


\section{Off-The-Shelf Parts}
The table below contains all the off-the-shelf parts required for building a complete HANDi Hand. Links are provided to particular vendors for convenience only; no affiliation between BLINCdev and the listed vendors exists or is implied.

%\begin{longtable}[l]{|l|l|l|l|l|l|l|l|l|l|}
%\caption{Off-the-Shelf Parts Specifications}
%\label{ots_parts_table}
%\hline
%\multicolumn{10}{|c|}{Sensors and Electronics}\\
%\hline
%\textbf{Item} & \textbf{Description} & \textbf{Vendor} & \textbf{Part No.} & \textbf{Link} & \textbf{QTY} & \textbf{Cost/Item} & \textbf{Ext. Cost} & \textbf{Curr.} & \textbf{Notes}
%\endfirsthead
%
%\hline
%\multicolumn{10}{|c|}{Sensors and Electronics}\\
%\hline
%\textbf{Item} & \textbf{Description} & \textbf{Vendor} & \textbf{Part No.} & \textbf{Link} & \textbf{QTY} & \textbf{Cost/Item} & \textbf{Ext. Cost} & \textbf{Curr.} & \textbf{Notes}
%\endhead
%
%\hline
%\endfoot
%
%\hline
%\endlastfoot
%
%\end{longtable}

\small{\begin{longtable}[l]{| p{1.5cm}|  p{2cm} |  p{1.5cm} |  p{1.5cm} |  p{2cm} |  p{1cm} |  p{1cm} |  p{1cm} |  p{1cm} |}
 \caption{Off-the-Shelf Parts Specifications.\label{ots_parts_table}}\\

 \hline 
 \textbf{Item} & \textbf{Description} & \textbf{Vendor} & \textbf{Part No.} & \textbf{Link} & \textbf{QTY} & \textbf{Cost / Item} & \textbf{Ext. Cost} & \textbf{Curr.}\\
 \hline
 \endfirsthead

 \hline
 \multicolumn{9}{|c|}{Continuation of Table \ref{long}}\\
 \hline
 \textbf{Item} & \textbf{Description} & \textbf{Vendor} & \textbf{Part No.} & \textbf{Link} & \textbf{QTY} & \textbf{Cost/Item} & \textbf{Ext. Cost} & \textbf{Curr.}\\
 \hline
 \endhead

 \hline
 \endfoot

 \hline
 \multicolumn{9}{| c |}{End of Table}\\
 \hline\hline
 \endlastfoot

\multicolumn{9}{| l |}{SENSORS AND ELECTRONICS}\\ \hline
Lots of lines & like this& like this& like this& like this& like this& like this& like this& like this\\
 Lots of lines & like this& like this& like this& like this& like this& like this& like this& like this\\
 Lots of lines & like this& like this& like this& like this& like this& like this& like this& like this\\
 Lots of lines & like this& like this& like this& like this& like this& like this& like this& like this\\
 Lots of lines & like this& like this& like this& like this& like this& like this& like this& like this\\
 Lots of lines & like this& like this& like this& like this& like this& like this& like this& like this\\
 Lots of lines & like this& like this& like this& like this& like this& like this& like this& like this\\
 Lots of lines & like this& like this& like this& like this& like this& like this& like this& like this\\
 Lots of lines & like this& like this& like this& like this& like this& like this& like this& like this\\
 Lots of lines & like this& like this& like this& like this& like this& like this& like this& like this\\
 Lots of lines & like this& like this& like this& like this& like this& like this& like this& like this\\
 Lots of lines & like this& like this& like this& like this& like this& like this& like this& like this\\
 Lots of lines & like this& like this& like this& like this& like this& like this& like this& like this\\
 Lots of lines & like this& like this& like this& like this& like this& like this& like this& like this\\
 Lots of lines & like this& like this& like this& like this& like this& like this& like this& like this\\
 Lots of lines & like this& like this& like this& like this& like this& like this& like this& like this\\

 \end{longtable}}

\section{Tools}
%% Assembly Flowchart

\chapter{Assembly Flowchart}  % Main chapter title

\label {Assembly Flowchart} 
%---------------------------------------------------------------------------------------------------------------------------------------
%
% ASSEMBLY FLOWCHART
%
%---------------------------------------------------------------------------------------------------------------------------------------
A particular order of operations must be followed when assembling the hand, and is outlined in the following flowchart. A process described in any bubble cannot be completed until all items attached to incoming arrows have been completed.
\begin{figure}
\centering
\includegraphics[width=\linewidth]{Figures/flowchart.png}
\caption{Flowchart depicting the order of operations in assembling the HANDi Hand.}
\label{fig:flowchart}
\end{figure}
%% Material Preparation

\chapter{Material Preparation}  % Main chapter title

\label {Material Preparation} 
%---------------------------------------------------------------------------------------------------------------------------------------
%
% MATERIAL PREPARATION
%
%---------------------------------------------------------------------------------------------------------------------------------------

\section{3D Printed Parts}
\textit{Estimated time 17h 0m}

\section{Potentiometers}
\textit{Estimated time 1h 45m}

In the current version, there are nine potentiometers (limited due to the number of analog input pins on the Arduino Mega). The suggested lengths of the connecting wires are found in Table \ref{potentiometer_table}.

\begin{table}[H]
\centering
\caption{Suggested wire lengths for potentiometers}
\label{potentiometer_table}
\begin{tabular}{| l | l |}
\hline
\textbf{Potentiometer ID} & \textbf{Suggested Wire Length}  \\ \hline
D0 & 150 mm \\ \hline
D1P & 150 mm \\ \hline
D1D & 190 mm \\ \hline
D2P & 150 mm \\ \hline
D2I & 190 mm \\ \hline
D3P & 150 mm \\ \hline
D3I & 190 mm \\ \hline
D4P & 150 mm \\ \hline
D5P & 150 mm \\ \hline
\end{tabular}
\end{table} 

Figure \ref{pot_pins} describes the polarity of the potentiometers. To prepare the potentiomters:

\begin{enumerate}
\item Cut off the top lone pin on the potentiometer as short as possible.
\item Solder the wires to the potentiometer and a straight pin header to the other end of each wire. See note below about D0.
\item Use black heat-shrink on the connections near the potentiometer, and coloured heat shrink on the far connections corresponding to the polarity noted in Figure 7.
\item Heat shrink the 3 wires together near the loose end using a light coloured heat shrink tubing. Label the wires here with the potentiometer ID.
\end{enumerate}
\textbf{Note:}  the straight pin headers for the D0 potentiometer must be soldered to the wires \textit{after} the potentiometer is installed in MC – Geared Rotator, and the wires routed through the narrow channel. Black heat-shrink tubing near the potentiometer will not fit in this channel, so it is omitted. 
\begin{figure}[H]
\centering
\includegraphics[width=0.3\linewidth]{Figures/Pot_Pins.png}
\caption{Potentiometer terminals diagram, viewed from the top (black side facing you). Pin \#1 is GND, pins labelled \#2 are SIG, and pin \#3 is VCC.}
\label{pot_pins}
\end{figure}


\section{FSRs}
\textit{Estimated time 1h 0m}

Each fingertip requires an FSR. The recommended wire length for finger FSRs is 250 mm; for the thumb FSR 200 mm is recommended. 
\begin{enumerate}
\item Solder each wire to the solder tabs of the FSR. 
\item Install FSR in fingertip (see 5.3.1 ).
\item Use black heat-shrink tubing on the connections near the FSR.
\item Solder the wires to straight pin headers.
\item Use yellow and red heat shrink tubing on the connections with the straight pin headers.
\item Heat shrink the two wires together using a light coloured heat shrink, and label the wires here with the ID of the finger they represent.
\end{enumerate}

\section{Screws}

%% Thumb Assembly

\chapter{Thumb Assembly}  % Main chapter title

\label {Thumb Assembly} 
%---------------------------------------------------------------------------------------------------------------------------------------
%
% THUMB ASSEMBLY
%
%---------------------------------------------------------------------------------------------------------------------------------------

\textit{Estimated time 1h 0m}

\section{Part Assembly}

\section{Mounting to Palm}
%% Finger Assembly

\chapter{Finger Assembly}  % Main chapter title

\label{Finger Assembly} 
%---------------------------------------------------------------------------------------------------------------------------------------
%
% FINGER ASSEMBLY
%
%---------------------------------------------------------------------------------------------------------------------------------------

\textit{Estimated time 0h 15m per finger}

%% Servo Installation and Finger Tensioning

\chapter{Servo Installation and Finger Tensioning}  % Main chapter title

\label{Servo Installation and Finger Tensioning} 
%---------------------------------------------------------------------------------------------------------------------------------------
%
% Servo Installation and Finger Tensioning
%
%---------------------------------------------------------------------------------------------------------------------------------------

\textit{Estimated time 0h 7m per finger}

%% Signal Routing Board

\chapter{Signal Routing Board}  % Main chapter title

\label{Signal Routing Board} 
%---------------------------------------------------------------------------------------------------------------------------------------
%
% SIGNAL ROUTING BOARD
%
%---------------------------------------------------------------------------------------------------------------------------------------

\textit{Estimated time 0h XXm}

%% Ventral Palm Cover and USB Webcam

\chapter{Ventral Palm Cover and USB Webcam}  % Main chapter title

\label{Ventral Palm Cover and USB Webcam} 
%---------------------------------------------------------------------------------------------------------------------------------------
%
% Ventral Palm Cover and USB Webcam
%
%---------------------------------------------------------------------------------------------------------------------------------------

\textit{Estimated time 0h 20m}

%% Palm Grips

\chapter{Palm Grips}  % Main chapter title

\label{Palm Grips} 
%---------------------------------------------------------------------------------------------------------------------------------------
%
% PALM GRIPS	
%
%---------------------------------------------------------------------------------------------------------------------------------------

\textit{Estimated time 1h 15m}

%% Wiring

\chapter{Wiring}  % Main chapter title

\label{Wiring} 
%---------------------------------------------------------------------------------------------------------------------------------------
%
% WIRING	
%
%---------------------------------------------------------------------------------------------------------------------------------------

\textit{Estimated time 1h 0m}



%----------------------------------------------------------------------------------------
%	MANUAL CONTENT - APPENDICES
%----------------------------------------------------------------------------------------

\appendix % Cue to tell LaTeX that the following "chapters" are Appendices
% Include the appendices of the thesis as separate files from the Appendices folder
% Uncomment the lines as you write the Appendices

% Appendix A

\chapter{Grip Pattern Template} % Main appendix title

\label{AppendixA}

The following page contains a template you can use to cut out the neoprene rubber grips. When printing this page, be sure to check in your print settings that you are using the original scale of the document.

%% Appendix B

\chapter{Grip Pattern Template} % Main appendix title

\label{AppendixB}

The following page contains a template you can use to cut out the neoprene rubber grips. When printing this page, be sure to check in your print settings that you are using the original scale of the document.

%\include{Appendices/AppendixC}


\end{document}  
